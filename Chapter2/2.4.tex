\documentclass{article}
\usepackage{amsmath,amssymb,amsthm,fancyhdr,enumerate,graphicx,array}
\usepackage[usenames,dvipsnames]{color}

\headheight=23pt
\textwidth=6.5in
\oddsidemargin=0in
\evensidemargin=0in
\textheight=8in
 \topmargin = -0.4 in


\theoremstyle{definition}
\newtheorem{ex}{Example}[subsection]
\newtheorem*{defn}{Definition}
\newtheorem*{note}{Note}
\newtheorem{exer}[ex]{Exercise}
\newtheorem*{thm}{Theorem}
\newtheorem*{props}{Properties}

\newcommand*{\dlim}{\displaystyle\lim}
\renewcommand{\u}[1]{\underline{#1}}
\renewcommand{\labelenumi}{(\alph{enumi})}
\renewcommand{\labelenumii}{(\roman{enumii})}


\newcommand{\lectureheader}{
\thispagestyle{fancy}
\lhead{Math 251: Calculus I\\ CRN 33502}
\rhead{University of Oregon\\ Spring 2017}
\chead{\textbf{Section \arabic{section}.\arabic{subsection}} \\ 
\textbf{\sectiontitle}}
\lfoot{} \cfoot{} \rfoot{}
}

%%%%%%%%%%%%%%%%%%%%%%%%%%%%%%%%%%%%%%%%%%%%%%%%%%%%%
\setcounter{section}{2}			%%%%%%%%%%%%%
\setcounter{subsection}{4}		%%%%%%%%%%%%%
\def\sectiontitle{Continuity}
%%%%%%%%%%%%%%%%%%%%%%%%%%%%%%%%%%%%%%%%%%%%%%%%%%%%%

\begin{document}
\lectureheader

\begin{defn}
A function $f(x)$ is said to be \u{continuous} at a point $a$ if 
$$\dlim_{x \to a} f(x) = f(a)$$
Notice that this definition implicitly requires three things for $f$ to be continuous at $a$:
 $f(a)$ and $\dlim_{x\to a}f(x)$ are both defined and are equal to each other.

We say that $f$ is continuous on an interval if it is continuous at every point in the interval. If $f$ is not continuous at $a$, we say that $f$ is discontinuous or that $f$ has a discontinuity at $a$.
%% For endpoints of the domain, we mean a one-sided limit
%%``can draw without picking up your pencil''
%% removable discontinuities, infinite discont (asymptote), jump discont
\end{defn}

\vspace{3in}

\begin{note}
Polynomials are continuous for all real numbers.
Rational functions are continuous on their domain (but could have discontinuities outside of their domain). Other functions that are continuous at every numbers in their domains include: root, trigonometric, inverse trigonometric, exponential, and logarithmic functions.
\end{note}

\begin{ex}
Is the function $$f(x)=\begin{cases} 4-x & \text{ if } x<1\\ \sqrt{x} & \text{ if } x\geq1\end{cases}$$ continuous on $(-\infty,\infty)$?
\end{ex}

\newpage

\begin{ex}
Find a value of $b$ such that $$g(x)=\begin{cases}
2bx & \text{ if } x\leq -1\\\
3x^2+x+b & \text{ if } x>-1\\ \end{cases} $$ is continuous for all real numbers.
\end{ex}
\vspace{2in}
\begin{props}
If $f$ is a continuous function and $\dlim_{x\to a}g(x)$ exists and is in the domain of $f$, then $$\dlim_{x\to a}f(g(x)) = f\left(\dlim_{x\to a}g(x)\right).$$
\end{props}

\begin{ex}
Find the indicated limit: $\dlim_{x\to 1}\ln\left(\dfrac{x^2-1}{x-1}\right)$
\end{ex}
\vspace{1.5in}
\begin{exer}
Recall the function from the Section 2.2 Handout: $$f(x)=\left\{\begin{tabular}{ll}
$\sin(x)$ & if $x<0$\\
$3$ & if $x=0$\\
$x^2$ & if $0<x\leq 1$\\
$1-x$ & if $x>1$
\end{tabular}
\right.$$
Sketch a graph of this function and use the graph to find all values $a$ for which $f$ is not continuous at $a$.
\end{exer}


\end{document}
